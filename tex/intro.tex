% ==============================================================================
%                                    DVG303
%                  Objektorienterad Design och Programmering
%                                Laboration #2
%
% Author:   Jonas Sjöberg
%           Högskolan i Gävle
%           tel12jsg@student.hig.se
%           https://github.com/jonasjberg
%
% License:  Creative Commons Attribution-NonCommercial-ShareAlike 4.0
%           International.  See LICENSE.md for full licensing information.
% ==============================================================================

\section*{Introduktion}\label{sec:intro}
\addcontentsline{toc}{section}{Introduktion}

\subsection*{Övergripande beskrivning}\label{sec:beskrivning}
\addcontentsline{toc}{subsection}{Övergripande beskrivning}
Det här är den andra av tre laborationer i objektorienterad design och
programmering. Ett fullständigt program kommer att utvecklas under
laborationerna. Processen kommer att innehålla många element av professionell
mjukvaruutveckling; design, dokumentation, revisionskontroll, etc., och syftar
till att utveckla praktiska färdigheter i mjukvaruutveckling.

\subsection*{Uppgifter}
\addcontentsline{toc}{subsection}{Uppgifter}
Instruktionerna är till viss del kopierade från källfilen
\texttt{oodp\_lab\_instruktioner\_ht15v4.pdf}.

\subsection*{Arbetsmetod}
\addcontentsline{toc}{subsection}{Arbetsmetod}
\begin{itemize}
    \item Koden skrivs i utvecklingsmiljön \texttt{Intellij IDEA} under
          \texttt{Linux 3.19.0-28-generic} och kompileras samt exekveras med 
          följande \texttt{Java}-version:
\begin{verbatim}
> $ java -version
java version "1.7.0_79"
OpenJDK Runtime Environment (IcedTea 2.5.6) (7u79-2.5.6-0ubuntu1.15.04.1)
OpenJDK Server VM (build 24.79-b02, mixed mode)
\end{verbatim}

    \item Rapporten skrivs i \LaTeX\ med texteditorn \texttt{Vim} och kompileras
          till pdf med \texttt{latexmk}.  \par Diagram och figurer skrivs i
          \texttt{PlantUML}-format och renderas med \texttt{Graphviz}. 
          Resultatet förhandsgranskas i realtid med hjälp av plugins i 
          \texttt{Intellij IDEA}.

    \item För revisionskontroll används \texttt{Git}.

    \item Analys av program under exekvering sker med hjälp av 
          \texttt{JIVE} -- Java Interactive Visualization Environment.
          \footnote{\url{http://www.cse.buffalo.edu/jive/}}

\end{itemize}

UML-diagram och dokumentation uppdateras löpande parallellt med källkoden.
Förändringar i källkoden har kommit att följa uppdaterade diagram likväl som
diagrammen har behövt uppdateras för att reflektera förändrad källkod eller
funktionalitet.
\par Varje uppgift finns representerad som en separat utvecklingsgren
(\emph{branch}) i \texttt{Git}. På så vis kan varje uppgift utvecklas oberoende
utan redundans och filduplicering. Det är också mycket enkelt att propagera
förändringar mellan grenar och individuella \emph{commits} med hjälp av
ändamålsriktiga ``diff''-verktyg. Ett sådant ingår i standarddistributioner av
\texttt{Git}, ett annant exempel är \texttt{Meld}.


\subsection*{Källkod}
\addcontentsline{toc}{subsection}{Källkod}
\par Källkod till programmet och rapporten finns att hämta på
\url{https://github.com/jonasjberg/DVG303_lab2}.
Hämta hem repon genom att exekvera följande från kommandoraden:
\begin{verbatim}
> $ git clone git@github.com:jonasjberg/DVG303_lab2.git
\end{verbatim}

