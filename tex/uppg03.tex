% ==============================================================================
%                                    DVG303
%                  Objektorienterad design och programmering
%                                Laboration #2
%
% Author:   Jonas Sjöberg
%           Högskolan i Gävle
%           tel12jsg@student.hig.se
%           https://github.com/jonasjberg
%
% License:  Creative Commons Attribution-NonCommercial-ShareAlike 4.0
%           International.  See LICENSE.md for full licensing information.
% ==============================================================================

\section{Uppgift 3}\label{sec:uppg3}

\subsection{}\label{sec:uppg3a}
\subsubsection*{Frågeställning}
När metoden \texttt{createPoint} anropas på ett objekt av typen
\texttt{FigureHandler}, så måste det skickas ett antal meddelanden mellan olika
objekt. Visa meddelanden med hjälp av ett sekvensdiagram!

\subsubsection*{Lösning}
Programmet analyserades med \texttt{JIVE} -- Java Interactive
Visualization Environment. \footnote{\url{http://www.cse.buffalo.edu/jive/}}

\texttt{JIVE} installeras som en plugin i \texttt{Eclipse Mars} version 4.5.0
och körs i ett interaktivt debug-``perspective''. Programmet kan generera olika
modeller av ett exekverande program;
\footnote{\url{http://www.cse.buffalo.edu/jive/presentations/cse505-10-12-01.pdf}}

\begin{enumerate}
\item Contour Model.
\item Object Diagram.
\item Sequence Model.
\item Sequence Diagram.
\item Event Log.
\end{enumerate}

En särskild klass vid namn \texttt{FigureHandlerTest} användes vid körning av
\texttt{JIVE}, som presenterar resultatet på olika sätt. Jag valde att
exportera som \texttt{csv}-textfil samt \texttt{png}-bild.

Sekvensdiagram för programmet \texttt{FigureHandlerTest} återfinns i
Figur~\ref{fig:sekv-point}, Figur~\ref{fig:sekv-line},
Figur~\ref{fig:sekv-rect} och Figur~\ref{fig:sekv-all}.
Resultatet av testerna är finns också bifogade i katalogen \texttt{diagram}.

% JIVE: Dynamic Analysis for Java
% Overview, Architecture, and Implementation
% Demian Lessa
% Computer Science and Engineering
% State University of New York, Buffalo
% http://www.cse.buffalo.edu/jive/presentations/cse505-10-12-01.pdf

% Jive
% Tool Overview
% April 07 :: Spring 2010
% Demian Lessa <dlessa@buffalo.edu>
% http://www.cse.buffalo.edu/jive/presentations/cse605-10-04-07a.pdf

\begin{figure}[ht]
\centering
\includegraphics[width=0.9\linewidth]{diagram/figureHandlerTest_All_Sequence-Diagram.png}
\caption{Sekvensdiagram för test av alla figurer.}
\label{fig:sekv-all}
\end{figure}

\begin{figure}[ht]
\centering
\includegraphics[width=\linewidth]{diagram/figureHandlerTest_Point_Sequence-Diagram.png}
\caption{Sekvensdiagram för \texttt{Point}.}
\label{fig:sekv-point}
\end{figure}

\begin{figure}[ht]
\centering
\includegraphics[width=0.8\linewidth]{diagram/figureHandlerTest_Line_Sequence-Diagram.png}
\caption{Sekvensdiagram för \texttt{Line}.}
\label{fig:sekv-line}
\end{figure}

\begin{figure}[ht]
\centering
\includegraphics[width=\linewidth]{diagram/figureHandlerTest_Rectangle_Sequence-Diagram.png}
\caption{Sekvensdiagram för \texttt{Rectangle}.}
\label{fig:sekv-rect}
\end{figure}

\begin{sidewaysfigure}
\centering
\includegraphics[width=\linewidth]{diagram/figureHandlerTest_All_Object-Diagram.png}
\caption{Objektdiagram för test av alla figurer.}
\label{fig:obj-all}
\end{sidewaysfigure}

\begin{figure}[ht]
\centering
\includegraphics[width=\linewidth]{diagram/figureHandlerTest_Point_Object-Diagram.png}
\caption{Objektdiagram för \texttt{Point}.}
\label{fig:obj-point}
\end{figure}

\begin{figure}[ht]
\centering
\includegraphics[width=\linewidth]{diagram/figureHandlerTest_Line_Object-Diagram.png}
\caption{Objektdiagram för \texttt{Line}.}
\label{fig:obj-line}
\end{figure}

\begin{figure}[ht]
\centering
\includegraphics[width=\linewidth]{diagram/figureHandlerTest_Rectangle_Object-Diagram.png}
\caption{Objektdiagram för \texttt{Rectangle}.}
\label{fig:obj-rect}
\end{figure}

\subsection{}\label{sec:uppg3b}
\subsubsection*{Frågeställning}
Skriv klasserna som implementerar interfacen i styrnings-API:et enligt figur 7.

\subsubsection*{Lösning}
Se bifogad källkod. Lösningen använder \texttt{StringBuilder} för att
konkatenera resultatet från superklassens \texttt{toString}-metod med klassens
egna data.


\subsection{}\label{sec:uppg3c}
\subsubsection*{Frågeställning}
Diskutera: Varför används typer som \texttt{List<FigureType>} i FigureHandler?
Kan man inte använda bara List? Eller en array?


\subsubsection*{Lösning}
Användningen av en \texttt{List} framför en vanlig array motiveras med att en
array inte kan ändra storlek dynamiskt. Storleken definieras vid skapande av
arrayen och är därefter fixerad. För det här användningsområdet är en
datastruktur med flexibla metoder för att ``sätta in'' och ``ta ut'' enskilda
element att föredra. En \texttt{List} har flera användbara funktioner som man
ofta måste skriva själv vid användning av arrays, ofta relaterade till sökning
och åtkomst.

Java version 1.5 introducerade ``typparametrisering''.
Det löser ett återkommande problem med listor och objekts typer. För en lista
som klarar av att lagra objekt av generell typ förlorar objekten sin specifika
typinformation då de stoppas in i listan. Då objektet ska plockas ut ur listan
måste de ``castas'' för att återfå sin ursprungliga typ. Ofta behövs särskilda
lösningar för att lagra objekt av olika typer, typparametriseringen erbjuder en
generell lösning genom att direkt i språket definera listor som lagrar en viss
typ av objekt.
Genom att skriva något i stil med \texttt{List<Figure>} specifieras vilken typ
av objekt listan kan innehålla.  Det fungerar även som en slags felkontroll där
kompilatorn ger varningar om ett inkompatibelt objekt stoppas in.


\javacode{tex/Generics.java}
\caption{Java SE Tutorials -- Learning the Java Language > Generics
\footnote{\url{https://docs.oracle.com/javase/tutorial/java/generics/why.html}}
}
\label{src:generics}


